%
% exposit.tex
%    システム制御情報学会  A4版クラスファイル
%    scitrans.cls のサンプル(解説・総説・展望記事のテンプレートファイル)
%
\documentclass[J]{scitrans}
\usepackage{graphicx}
%\usepackage[whole]{bxcjkjatype}
%
%               Usage:  \documentclass[J]{scitrans} (和文の場合)
%                       \documentclass[E]{scitrans} (英文の場合)
%
% 年, 巻, 号 ページの設定
%
\UseRawInputEncoding
\appearyear{xxx}
\vol{xxx}
\numberinvol{xx}
\setcounter{page}{1}
\setcounter{volumepage}{1}

\Journal{解説}{『特集名』}{\expositry}      %% 解説の時
%%\Journal{総説}{『特集名』}{\survey}       %% 総説の時
%%\Journal{展望}{『特集名』}{\technicalview}%%  展望の時

%%\Journal{カテゴリ}{}      %% 特集号でない時は, 第2引数を空に

%%\ForSubmission            %% 提出前に一度コメントをはずして
                            %% 英文題目および著者名をご確認ください
\def\ddash{\rule[0.33zh]{2zw}{.03zh}}
\begin{document}

\title{埋め込み法が拓くネットワーク科学の新展開}
\author{幸若 完壮*}

\etitle{A New Paradigm for Network Science with Graph Embeddings}
\eauthor{Sadamori {\sc Kojaku}*}
\headingtitle{埋め込み法が切り拓くネットワーク分析の新展開}
\headingauthors{システム}

\maketitle

\acceptdate{xxx年xx月xx日}

\address{*}{Luddy School of Informatics, Computing, and Engineering, Indiana University Bloomington, USA}

\keywords{ネットワーク埋め込み法, 科学の科学, 機械学習}

%%% つぎの \input を削除し,本文を書き出して下さい.
%-----------------
\section{はじめに}
%-----------------
\label{sec:introduction}

友好関係, 銀行の融資関係, 食物連鎖など社会や自然界のあらゆる場所にネットワークは存在する。
ネットワークは人間や銀行といった「モノ」と友好関係や取引関係といった「つながり」からなり, 
つながりを介したモノ同士の相互作用が感染の拡大や倒産の連鎖, 異常繁殖といった突発的で大規模な現象の一因となっている\cite{}。
%ネットワークはモノ(頂点)とモノのつながり(枝)の集まりである。
%これらのつながりを介してモノとモノがお互いに影響しあい, 感染症の拡散や倒産の連鎖, 生物の異常繁殖といった突発的で大規模な現象が引き起こる\cite{}。
%5例えば, 感染症は多くの人と接触機会を持つ人に伝わりやすく, その人を介してさらに多くの人に伝わることで爆発的な感染の連鎖が起こる\cite{}。

デジタル技術の進歩によって様々なネットワークがデータとして記録される一方で, その分析には超えるべき大きな障壁がある。
それは, ネットワークが単なる「つながりの集まり」ではないため, そのままの形で一般的なデータ分析法を適用しても十分な結果が得られない、という障壁である。
例えば, 感染を避けるために2人の人間が対面接触を避けたとしても, 2人の共通の友人と対面接触を続けた場合は感染のリスクが下がらない。
反対に, 共通の友人を持たない, あるいは友人らとの対面接触も同時に避ける場合, 感染の抑制効果が期待できる。
このようにつながりの組み合せが重要な意味を持つため, ネットワーク分析のための専用の手法や理論が必要となる。

この障壁を解決する方法として, ネットワークをベクトルに圧縮変換する技術\ddash 埋め込み法\ddash が登場し, 新たな展開を見せている。
埋め込み法は, 頂点をある空間上の点として捉えて, 頂点の位置関係によってその複雑な関係を表現する手法である(図1)。
各頂点は位置を表すベクトルで表現されるため, ベクトルデータを入力とする様々な統計的・機械学習的なデータ分析法がネットワーク分析に利用可能となる。

本稿では, 埋め込み法が拓くネットワーク科学の新たな展開について概説する。
具体的に, 前半部では埋め込み法の代表的な手法とその応用例を紹介し基本的な考えと仕組みを説明する。
後半部では, 埋め込み法の応用事例として科学雑誌の引用ネットワークから学術分野のマップを作成した事例を紹介する。

\section{ネットワーク埋め込み法}

- なぜ埋め込みか
    - NLPの要請に触れておく
- どうやって埋め込む?
   - Neural Netで図的な解説
    - 情報圧縮と解凍
- 自然言語処理での成功例
    - Analogy test
    - Material2Vec
- Networkの埋め込み
    - Grapから文の生成
- Airport netでの応用

\section{科学の地図を作る}

- Journal2Vec
- Translational axis




%-----------------
\section{おわりに}
%-----------------


% 謝辞
%------------------
\acknowledgement
%-----------------



\authorbiography{幸,若, ,完,壮}{こう,じゃく,,さだ,もり}{非正会員}{%
 2015年9月××大学大学院工学研究科○○工学専攻△△課程修了.
 同年X月××助手.19XX年X月××となり現在に至る.
 ××の研究に従事.
 ××などの会員.}

%\authorbiography{佐,藤, ,花,子}{さ,とう,,はな,こ}{正会員}{%
%  19XX年X月××大学大学院工学研究科○○工学専攻△△課程修了.
%  同年X月××助手.19XX年X月××となり現在に至る.
%  ××の研究に従事.
%  ××などの会員.}

%\authorbiography{田,中, ,次,郎}{た,なか,,じ,ろう}{正会員}{%
%  19XX年X月××大学大学院工学研究科○○工学専攻△△課程修了.
%  同年X月××助手.19XX年X月××となり現在に至る.
%  ××の研究に従事.
%  ××などの会員.}

\end{document}
%
%% end of exposit.tex
