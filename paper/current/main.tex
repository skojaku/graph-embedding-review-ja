%
% exposit.tex
%    システム制御情報学会  A4版クラスファイル
%    scitrans.cls のサンプル(解説・総説・展望記事のテンプレートファイル)
%
\documentclass[J]{scitrans}
\usepackage{graphicx}
%\usepackage[whole]{bxcjkjatype}
%
%               Usage:  \documentclass[J]{scitrans} (和文の場合)
%                       \documentclass[E]{scitrans} (英文の場合)
%
% 年, 巻, 号 ページの設定
%
\UseRawInputEncoding
\appearyear{xxx}
\vol{xxx}
\numberinvol{xx}
\setcounter{page}{1}
\setcounter{volumepage}{1}

\Journal{解説}{『特集名』}{\expositry}      %% 解説の時
%%\Journal{総説}{『特集名』}{\survey}       %% 総説の時
%%\Journal{展望}{『特集名』}{\technicalview}%%  展望の時

%%\Journal{カテゴリ}{}      %% 特集号でない時は, 第2引数を空に

%%\ForSubmission            %% 提出前に一度コメントをはずして
                            %% 英文題目および著者名をご確認ください

\begin{document}

\title{埋め込み法が拓くネットワーク科学の新展開}
\author{幸若 完壮*}

\etitle{Usage of Style File for Transactions of the Institute of
        Systems, Control and Information Engineers}
\eauthor{Sadamori {\sc Kojaku}*}
\headingtitle{埋め込み法が拓くネットワーク分析の新展開}
\headingauthors{システム}

\maketitle

\acceptdate{xxx年xx月xx日}

\address{*}{Luddy School of Informatics, Computing, and Engineering, Indiana University Bloomington, USA}

\keywords{ネットワーク埋め込み法, 科学の科学, 機械学習}

%%% つぎの \input を削除し,本文を書き出して下さい.
%-----------------
\section{はじめに}
%-----------------
\label{sec:introduction}


- ネットワークは身近にある
- 扱いにくさとベクトル化による解決
- 内容の見出し


\section{ネットワーク埋め込み法}

- なぜ埋め込みか
    - NLPの要請に触れておく
- どうやって埋め込む?
   - Neural Netで図的な解説
    - 情報圧縮と解凍
- 自然言語処理での成功例
    - Analogy test
    - Material2Vec
- Networkの埋め込み
    - Grapから文の生成
- Airport netでの応用

\section{科学の地図を作る}

- Journal2Vec
- Translational axis




%-----------------
\section{おわりに}
%-----------------


% 謝辞
%------------------
\acknowledgement
%-----------------



\authorbiography{幸,若, ,完,壮}{こう,じゃく,,さだ,もり}{非正会員}{%
 2015年9月××大学大学院工学研究科○○工学専攻△△課程修了.
 同年X月××助手.19XX年X月××となり現在に至る.
 ××の研究に従事.
 ××などの会員.}

%\authorbiography{佐,藤, ,花,子}{さ,とう,,はな,こ}{正会員}{%
%  19XX年X月××大学大学院工学研究科○○工学専攻△△課程修了.
%  同年X月××助手.19XX年X月××となり現在に至る.
%  ××の研究に従事.
%  ××などの会員.}

%\authorbiography{田,中, ,次,郎}{た,なか,,じ,ろう}{正会員}{%
%  19XX年X月××大学大学院工学研究科○○工学専攻△△課程修了.
%  同年X月××助手.19XX年X月××となり現在に至る.
%  ××の研究に従事.
%  ××などの会員.}

\end{document}
%
%% end of exposit.tex
