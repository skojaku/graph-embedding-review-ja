%
% exposit.tex
%    システム制御情報学会  A4版クラスファイル
%    scitrans.cls のサンプル(解説・総説・展望記事のテンプレートファイル)
%
\documentclass[J]{scitrans}
\usepackage[dvipdfmx]{graphicx}
\graphicspath{{../../figs/}}

\usepackage{amsmath}
%\usepackage[whole]{bxcjkjatype}
%
%               Usage:  \documentclass[J]{scitrans} (和文の場合)
%                       \documentclass[E]{scitrans} (英文の場合)
%
% 年, 巻, 号 ページの設定
%
\UseRawInputEncoding
\appearyear{xxx}
\vol{xxx}
\numberinvol{xx}
\setcounter{page}{1}
\setcounter{volumepage}{1}

\Journal{解説}{『特集名』}{\expositry}      %% 解説の時
%%\Journal{総説}{『特集名』}{\survey}       %% 総説の時
%%\Journal{展望}{『特集名』}{\technicalview}%%  展望の時

%%\Journal{カテゴリ}{}      %% 特集号でない時は, 第2引数を空に

%%\ForSubmission            %% 提出前に一度コメントをはずして
                            %% 英文題目および著者名をご確認ください
\def\ddash{\rule[0.33zh]{2zw}{.03zh}}
\def\legen#1{{\large \bf #1}}
\begin{document}

\title{埋め込み法が拓くネットワーク科学の新展開}
\author{幸若 完壮*}

\etitle{A New Paradigm for Network Science with Graph Embeddings}
\eauthor{Sadamori {\sc Kojaku}*}
\headingtitle{埋め込み法が切り拓くネットワーク分析の新展開}
\headingauthors{システム}

\maketitle

\acceptdate{xxx年xx月xx日}

\address{*}{Luddy School of Informatics, Computing, and Engineering, Indiana University Bloomington, USA}

\keywords{ネットワーク埋め込み法, 科学の科学, 機械学習}

%%% つぎの \input を削除し,本文を書き出して下さい.
%-----------------
\section{はじめに}
%-----------------
\label{sec:introduction}

友好関係、 銀行の融資関係、食物連鎖など社会や自然界のあらゆる場所にネットワークは存在する。
ネットワークは人間や銀行といった「モノ」と友好関係や取引関係といった「つながり」からなり、 
モノがお互いに影響を与え合うことで突発的で大規模な現象、例えば感染の拡大や倒産の連鎖、異常繁殖などの一因となっている\cite{}。
%ネットワークはモノ(頂点)とモノのつながり(枝)の集まりである。
%これらのつながりを介してモノとモノがお互いに影響しあい、 感染症の拡散や倒産の連鎖、生物の異常繁殖といった突発的で大規模な現象が引き起こる\cite{}。
%5例えば、 感染症は多くの人と接触機会を持つ人に伝わりやすく、その人を介してさらに多くの人に伝わることで爆発的な感染の連鎖が起こる\cite{}。

デジタル技術の進歩によって様々なネットワークがデータとして記録される一方で、その分析には超えるべき大きな障壁がある。
それは、ネットワークはモノやつながりの集まりではないため、一般的な手法でデータ分析しても十分な結果が得られないという障壁である。
例えば、ある感染症の感染疑いがある人を、人と人の接触データから見つける場合、感染者と接触した人を調べるだけでは不十分である。
なぜなら直接の接触がなくても、例えば共通の友人を介して感染することや、友人の友人、さらにその友人を介して感染する可能性もある。
そのため、感染疑いのある人を発見するためには個々の接触だけでなく、ネットワーク全体の接触関係を考えなくてはならない。
このように、ネットワークではつながりの組み合せが重要な意味を持つため、 ネットワーク分析のための専用の手法や理論が必要となる。

この障壁を解決する方法として、ネットワークをベクトルに圧縮変換する技術\ddash 埋め込み法\ddash が登場し、新たな展開を見せている。
埋め込み法は、 頂点をある空間上の点として捉えて、頂点の位置関係によってその複雑な関係を表現する手法である(図1)。
各頂点は位置を表すベクトルで表現されるため、ベクトルデータを入力とする様々な統計的・機械学習的なデータ分析法がネットワーク分析に利用可能となる。

本稿では、 埋め込み法が拓くネットワーク科学の新たな展開について概説する。
具体的に、 前半部では埋め込み法の代表的な手法とその応用例を紹介し基本的な考えと仕組みを説明する。
後半部では、 埋め込み法の応用事例として科学雑誌の引用ネットワークから学術分野のマップを作成した事例を紹介する。

\section{埋め込み法}

\subsection{自然言語処理と埋め込み法}

ネットワーク埋め込み法の元になっている埋め込み法は、もともと自然言語処理のために開発された手法である。
自然言語処理では、曖昧な人間の言語を機械で処理できる形にどのように変換するかが大きな課題の一つであった。
率直に考えれば、1つの単語に対して1つの数字を割り当てれば一応は数値化できる。
例えば、「値段は高いが質が低い料理」という文を「値段,高い,質,低い,料理」と区切って各単語に1つの数字を割り当てて「1,2,3,4,5」と表現する。
これは局所表現と呼ばれる単語の表現方法であるが、「高いの対義語は低いである」といった単語の関係を捉えられない。


\begin{figure}
    \centering
    \includegraphics[width=0.8\hsize]{baseball-map.pdf}
    \caption{
        プロ野球球団と都道府県の分散表現。
        可視化のために各単語の空間ベクトルの次元(300次元)を主成分分析によって2次元に減らした。
        点線は各球団と本拠地がある都道府県を示している。
        都道府県と球団が左右に分かれ、球団の上下の順序は本拠地のそれと大まかに対応している。
        単語の位置を得るため、国立国語研究所の日本語ウェブコーパスで学習させたWord2Vecを用いた\protect\cite{kawamura2020chive}。
    }
    \label{fig:baseballmap}
\end{figure}

この局所表現の欠点を解決したのが「分散表現」だ。
分散表現は、各単語をベクトル空間上の点と捉え、単語同士の意味的な関係を空間上の位置関係で表現する。
例えば、プロ野球球団と都道府県名を分散表現で記述しよう(図1)。球団同士、都道府県同士を近くに配置することで「球団」と「都道府県」の区別を表現できる。
さらに、単語の位置べクトルを直接演算することで様々な推論をすることができる。
例えば、「北海道」と「ファイターズ」という関係は
\begin{align}
    v({\rm 北海道}) + v({\rm 球団}) = v({\rm ファイターズ})
\end{align}
と書くことができる。 ここで単語$w$のベクトルを$v(w)$とした。
したがって「福岡の球団は?」という問いの答えは
$v({\rm 福岡}) + v({\rm 球団}) = v({\rm 福岡}) + v({\rm ファイターズ}) - v({\rm 北海道})$で与えられ、
この答えが$v({\rm ホークス})$に大体一致する。

分散表現は膨大な単語の配置を決定する難しさや計算量が大きいといった実用上の課題のため実践が進まなかった。
これらの課題を解決し、自然言語処理に大きな進展をもたらした手法が「Word2Vec」である。
図1の分散表現もWord2Vecによって得られたものである。

\subsection{Word2Vec}

\begin{figure}
    \centering
    \includegraphics[width=\hsize]{schematic-word2vec.pdf}
    \caption{
        Word2Vecは単語を入力としてその周辺の単語を出力するニューラルネットワークである。
        中間層が入力する単語の埋め込み空間上の座標である。
    }
    \label{fig:word2vec}
\end{figure}

Word2Vecは2層で構成されるニューラルネットワークである(図\ref{fig:word2vec})。
入力層は文章中の単語、出力層は周辺に現われる単語、そして中間層が入力単語の分散表現である\footnote{これはSkip-gram法と呼ばれるWord2Vecの1つの手法である。Word2Vecの別の手法として、各単語を出力、周辺語を入力とするCBOW(Continuous Bag-of-Words)法がある\cite{Mikolov2013}。}。

この手法の重要なポイントは出力層である周辺の単語である。
周辺の単語が近しければ、中間層である分散表現も近しい。言い換えれば、Word2Vecで得られる分散表現では、使われる文脈が似ている単語は近くに配置されるのである。
例えば、図1の球団名は似たような文脈で使われるため近くに配置されるのである。

Word2Vecは情報圧縮装置と見ることもできる。
Word2Vecの入力は単語であるが、実際には単語の記号の代わりに高次元のベクトルを入力する。
このベクトルでは1つの要素が``1''で他の全てが要素が``0''であり、``1''が出現する場所で入力単語を表現する。
そのため、単語が$N$個あれば入力ベクトルの長さは$N$である。
この高次元ベクトルは入力層から中間層に渡ってより小さな次元のベクトルに「圧縮」され、その後に中間層から出力層に渡って文脈を表すベクトルに変換される。
この圧縮されたベクトルが入力単語の分散表現である。

Word2Vecの興味深い活用例を1つ紹介しよう。
研究\cite{Tshitoyan2019}では材料科学に関する約330万本の論文の要旨をWord2Vecに学習させ、様々な材料の分散表現を構築した。
この「材料空間」は、様々な物性、例えば元素の周期表、物質の強磁性、熱電特性、結晶構造の類似性などを捉えている。
また、材料の位置ベクトルを演算して、$v({\rm AI}_2{\rm O}_3) - v({\rm AI}) + v({\rm Si}) = v({\rm SiO}_2)$といった物質の反応や、
$v({\rm 二重六方最密構造}) - v({\rm La}) + v({\rm Cr}) = v({\rm 体心立方格子構造})$といった結晶構造の関係を推論することができる。
さらに、Word2Vecは過去の論文の要旨から当時は未発見の材料の特性を、2から4割程度の精度で正しく予言したのである。

Word2Vecの登場によって自然言語処理は大きく進展した。
Word2Vecはその後様々な改良が加えられ\cite{Levy2014,pennington-etal-2014-glove,joulin2016fasttext,Bojanowski2017}、無償で利用できるパッケージにまとまっている\cite{gensim}。

\subsection{グラフ埋め込み法}

Word2Vecなどの自然言語の埋め込み法を応用して頂点を埋め込む手法が、これから紹介する「グラフ埋め込み法」である。

ネットワークの頂点の関係は、関係が未知な単語と異なり、枝で明確に表現されている。
ではネットワークに埋め込み法を適用すると何が嬉しいのだろうか?
頂点は直接つながっている頂点と関係を持つと同時に、直接つながっていない頂点とも関係を持つ。
例えば「敵の敵は味方」ということわざがある。自分(頂点)にとって直接的な敵対関係(枝)にある敵と敵対関係にある別の誰かは、たとえ自分と直接的な関係がなくても、味方という間接的な関係を持つ。
ある頂点と枝でつながる頂点はネットワーク全体から見ると少数である。
一方で、直接つながっていない頂点は数多くあり、それら頂点との関係は単語の関係と同じように非自明である。
グラフ埋め込み法は、頂点の関係を枝ではなく空間上の位置関係で表現する手法である。
枝で直接つながっていない頂点対であっても、配置されている距離や方向から、関係の強さや種類を調べることができる。
そしてこれらの分析は簡単なベクトル演算で行うことができる。

ではどうやって頂点を埋め込むのだろうか?
ここではネットワーク埋め込み法の先駆けであるDeepWalkについて紹介する。
自然言語処理で発展した埋め込み法は、文章が入力であるので、当然ながらネットワークを入力として受け付けない。
この問題を解決し、ネットワークに自然言語処理の埋め込み法を適用できるようにしたのがDeepWalk\cite{Bryan2014}である。

\begin{figure}
    \centering
    \includegraphics[width=0.8\hsize]{schematic-deepwalk.pdf}
    \caption{
    }
    \label{fig:deepwalk}
\end{figure}

DeepWalkでは、各頂点を1つの単語として、頂点の列(文)をネットワークから生成する(図\ref{fig:deepwalk})。
生成する文がある頂点$i$から始まるとしよう。DeepWalkでは、$i$の次に来る頂点を隣接する頂点の中から確率的に選ぶ。
これを繰り返し行うことで、頂点を単語とする文を生成する。
あとはこの生成された文をWord2Vecに与えれば、頂点の埋め込みが得られる。
この「ある頂点から別の頂点を確率的に選んで移動する運動」はネットワークにおけるランダム・ウォークと呼ばれる。
ランダム・ウォークには様々な種類があり、別種のランダム・ウォークを利用したネットワーク埋め込み法も提案されている\cite{Grover2016,Dong2017}。

\subsection{埋め込み空間の分析ツール}

埋め込み空間は高次元であり、座標を見て頂点の位置関係を直接理解することは難しい。
ここでは埋め込み空間の可視化と分析のためのツールについて紹介する。

埋め込み法を用いた研究では、頂点を高次元の埋め込み空間から2次元平面に射影して可視化することがよく行われる。
よく用いられる手法として、線形の次元削減法である主成分分析法や、非線形の次元削減法であるt-SNE\cite{Maaten2008}, U-map\cite{McInnes2018}などがある。
本稿では線形判別分析(Linear Disciminative Analysis; LDA)を主として用いる。
LDAは教師あり学習の手法で、ラベルが付与されたデータ点を入力として、ラベルが異なるデータ点同士がなるべく離れるような低次元空間へ射影する。
LDAは、頂点のラベルと埋め込み座標の定性的な関連を調べるときに有用である。
どの次元削減法も頂点の元々の位置を変えるため、結果の解釈には注意しなければならい。
例えば、ある2頂点が次元削除した空間で距離が近いからといって、元々の空間で近いとは限らない。
したがって、次元削除法による可視化は分析の方針を立てる手段として用いることをおすすめする。

頂点のメタ情報を使うことで埋め込み空間を解釈する方法がある。
例えば、単語の埋め込み空間で、肯定的な単語から否定的な単語に軸を設定することで、単語の肯定度合いを測る軸を作ることができる。
これはSemAxisと呼ばれる手法である~\cite{An2018}。
軸の作り方は1つではなく、次元削減法を使って作ることもできる。
例えば、肯定的な単語集合をグループA、否定的な単語集合をグループBとし、LDAを使って1次元へ次元削減する。
これによって2グループの単語が極力混ざらない軸を作ることができる。

\section{ケーススタディー}

グラフ埋め込み法で実際にネットワークを埋め込んでみよう。
本稿では航空網と雑誌の引用ネットワークの2つのネットワークを埋め込んでいく。
また、埋め込み空間を分析するために便利な解析手法についても紹介していく。
本稿で用いたコードは\cite{}で公開している。

\subsection{空港ネットワークの埋め込み}

\begin{figure}
    \centering
    \includegraphics[width=\hsize]{embedding-airport-annotation.png}
    \caption{
        空港ネットワークの埋め込み。可視化のため、128次元の埋め込み空間をLDAを用いて2次元に射影した。
        各点は空港を表し、点の大きさはネットワーク上で隣接する空港の数(次数)を表す。
        LDAへの入力ラベルとして、空港の位置する地域(アジア、アフリカ、アメリカ、オセアニア、ヨーロッパ)を用いた。
    }
    \label{fig:airport}
\end{figure}

空港ネットワークは、地球表面に埋め込まれたネットワークである。
空港をグラフ埋め込み法を用いて配置しなおしたとき、元々の空港の位置関係が復元されるのか?されなければどのような違いがあるか見てみよう。

%空港同士のつながりは空港の位置に影響される一方で、空港が位置する国の社会的、経済的な関係にも影響される~\cite{Guimera2005}。
Openflight.orgで公開されているデータ\cite{opsahl_2011}を用いて、空港を頂点、定期便の有無を枝で表した空港ネットワークを構築した。
枝に方向はなく重みもない。このネットワークを、行列分解法\cite{Qiu2018}に基づくDeepWalkを用いて128次元の空間に埋め込んだ。


埋め込み空間では、空港の地理的な関係に加えて社会的・経済的な結びつきが表れている(図\ref{fig:airport})。
埋め込み空間には、大まかに3つのクラスターがあり、それぞれアジア・アフリカ・ヨーロッパ地域、アメリカ地域、オセアニア地域の空港からなる。
アジア・アフリカ・ヨーロッパ地域のクラスター内では、アジアとヨーロッパ地域の空港はあまり混ざっていないのに対して、歴史的な理由で社会的・経済的な結びつきが強いアフリカとヨーロッパ地域の空港は混ざっている。
他地域でも地理的な近さだけでは説明できない空港の結びつきが表れている。
例えば、地理的にアジアに近いパプアニューギニアの空港(Port Moresby Jacksons Intl)よりも、オーストラリアの空港(Melbourne Intl, Syndney Intl)がアジア地域のクラスターに近い。
同様にして、アメリカ地域では、地理的にヨーロッパに近いMiami空港より、John F Kennedy IntlやNewwark Liberty Intlがヨーロッパ地域寄りに埋め込まれている。
各地域クラスターの境界には、多地域を結ぶハブ空港(Frankfurt Main, Heathrow, John F Kennedy Intl)や地域間を接続するゲートウェイ空港(Guam Intl, Honolulu Intl, Narita Intl)があり、
空港の位置関係から空港の役割を読み解くことができる。

\subsection{ライフサイエンス研究の埋め込み}

新型コロナウイルス(COVID-19)の流行によって、社会・経済的な活動の制限が長期間続いており、
治療法やワクチンの開発が精力的に進められている。

治療法やワクチン開発には、様々な研究のプロセスを段階的に経る必要がある。
例えば、ワクチン開発には、ウイルスの性質を調べ上げ、ワクチンの候補となるタンパク質を発見する必要がある。
次に、候補となるタンパク質からワクチンを生成し、多くの動物実験や臨床試験を行って安全性を確認しなければならない。
さらに、開発したワクチンの生産の効率化や、市民に届けるための社会的な仕組みの構築などが必要である。
この基礎研究から臨床応用までの段階的なプロセスは``bench-to-bedside''、つまり実験台(bench)から患者の枕元(bedside)までのプロセスと呼ばれている。

Bench-to-bedsideのプロセスは優れた基礎研究の成果を実用化するために重要である。
しかし、多くの基礎研究の成果が臨床応用につながっていない問題や、臨床応用までに時間がかかる問題が指摘されており、プロセスの効率化が求められている\cite{Khoury2007}。
では、基礎研究が臨床応用にどのように橋渡しされているのだろうか?
橋渡しの工程の中で、どの研究領域がどの程度重なっているのか?断絶はないか?
これらの問いを、埋め込み法で答えてみよう。

\begin{figure*}[h!]
    \centering
    \begin{minipage}{\hsize}
        \legen{A} \\
        \includegraphics[width=\hsize]{medical-journal-map.png} 
    \end{minipage}
    \begin{tabular}{cc}
        \begin{minipage}{0.5\hsize}
            \legen{B} \\
            \includegraphics[width=\hsize]{bench-to-bedside.pdf}
        \end{minipage}
        &
        \begin{minipage}{0.45\hsize}
            \legen{C} \\
            \includegraphics[width=\hsize]{ta-citation.png} 
        \end{minipage}
    \end{tabular}
    \caption{
        生命科学分野の雑誌の引用ネットワークの埋め込み。
        MEDLINEに登録されている$5,274$誌を128次元の空間に埋め込んだ。
        {\bf (A)} 可視化のためLDAを用いて2次元面に射影した。LDAに与えるデータ点(雑誌)のラベルとして、National Library of Medicine (NLM)による
        雑誌の分野別分類を用いた。
        {\bf (B)} 基礎から臨床までのスペクトル。
        このスペクトルは、基礎研究であるMicrobiologyと、臨床研究であるNursingとChildがなるべく離れるように、埋め込み空間上の全雑誌を直線に射影したものである。
        上から下にかけて、軸上の雑誌の位置の中央値で分野を降順に並べている。
        {\bf (C)} 基礎と臨床研究の引用ネットワーク。 全雑誌を(B)の軸上の位置で順にならべ、全雑誌を10のグループに等分した。
        黒円は雑誌のグループを表す。グループ$i$から$j$への矢印は$i$から$j$への引用を表す。
        矢印の幅は引用の数を表し、青は臨床から基礎研究の方向への引用、オレンジはその逆向きの方向への引用を表す。
    }
    \label{fig:journal-map}
\end{figure*}

優れた基礎研究の成果は、臨床研究の土台として引用される~\cite{Weber2013}。
この考えのもと、論文雑誌の引用ネットワークを埋め込み、埋め込んだ空間から基礎研究から臨床応用へのスペクトルがあるか確認する。
論文雑誌の引用ネットワークを構築するために、ライフサイエンスの文献情報を収録したMEDLINEを用いる~\cite{MEDLINE}。
MEDLINEには$5,274$の雑誌が登録されており、専門家によって雑誌が分野別に分類されている。
これらの雑誌が出版した全ての論文の書誌情報から雑誌の引用ネットワークを構築した。

構築した引用ネットワークをDeepWalkを用いて128次元の空間に埋め込んだ~(図\ref{fig:journal-map}A)。
全体的に、同じ分野の雑誌が大体まとまって配置され、分野の位置から基礎研究と臨床研究へのスペクトルを見ることができる。
具体的に、基礎研究である生物学(Biology)系の分野から始まり、薬学(Pharmacology, Medicine)や腫瘍学(Neoplasms), そして公衆衛生(Public Health)へと分野が連続して重なっている。
臨床研究が多い小児科学(Child)や看護学(Nursing)は、公衆衛生と近いが重なりは比較的少ない。
 
基礎研究から臨床研究までのスペクトルをより定量的に求めよう。
Microbiologyの雑誌を基礎研究グループ、小児科学と看護学の雑誌を臨床研究グループとし、2グループが極力混ざらないような軸をLDAによって求めた(図\ref{fig:journal-map}B)。
この軸を基礎-臨床軸と呼ぶ。
この軸上では、雑誌が基礎から臨床研究まで順に並んでいる。
この順序は、専門家による4雑誌(Biological Chemistry, JCI, NEJM, JAMA)の基礎から臨床までの順序付けと一致している\cite{Narin1976}。
軸の両端の分野(Microbiology, Child, Nursing)を橋渡しする分野には、一般科学、薬学、腫瘍学、公衆衛生がある。
%この軸上では、軸設定に用いた基礎研究グループ(Microbiology)と, 臨床研究グループ(Child, Nursing)の雑誌がほぼ混ざらずに両端に配置されていることから、狙い通りに軸が設定されていることが確認できる。

基礎から臨床の各段階の研究はどの段階の研究成果を土台にしているのだろうか?
この問いに答えるため、雑誌を基礎-臨床軸の位置で並び替えて10のグループに等分し、グループ間の引用数を数えた(図\ref{journal-map}C)。
全体的に、各グループは基礎-臨床軸上で近しいグループの雑誌を多く引用している。
研究の段階が臨床研究に進むにつれてグループ間の引用が少なくなっているが、これは分野の慣習によるところがある。
基礎-臨床軸上で、基礎から臨床グループへの引用($365,898,827$)は、臨床から基礎への引用($113,195,427$)の約3倍である。
したがって、各段階の研究は少し基礎よりの研究を引用する傾向があるが、臨床から基礎への研究の流れも少なからずあることがわかる。

%-----------------
\section{おわりに}
%-----------------

ネットワークをコンパクトなベクトルに変換する手法であるネットワーク埋め込み法を紹介した。
ネットワークをベクトルに変換することで、一般的な機械学習法や統計解析法を使ったネットワークの分析が可能となる。
また、埋め込み空間における頂点の位置関係から、頂点の役割やネットワークにおける立ち位置を読み解くことができる。
航空網と雑誌の引用関係の分析を通して、次元削除法やSemAxisなどの埋め込み空間の理解を助けるツールを紹介した。

本稿ではWord2VecをベースにしたDeepWalkを紹介した。
この他にも、LINE\cite{Tang:2015}、Node2Vec\cite{Grover:2016}、PTE\cite{Tang2015}などのネットワーク埋め込み法が広く使われている。 
DeepWalkは一般のネットワークを埋め込む手法であるが、特定の種類のネットワーク、例えば、2部グラフを埋め込む方法\cite{Gao2018}や多重ネットワーク(Multiplex networks)を埋め込む方法\cite{Dong2017}も提案されている。

埋め込み法の実践例として、航空網と雑誌の引用ネットワークを埋め込んで分析した。
分析に用いたコードは\cite{code}から入手することができる。
雑誌の引用関係の分析に関して、様々な分野の研究活動を埋め込み法で分析した研究\cite{Peng2020}を参考にし、本稿では生命科学の分野に焦点を絞って分析した。
埋め込み法を使って研究活動を分析する試みとして、論文のキーワードから学術用語を埋め込んだ研究\cite{Chinazzi2019,Ke2019}や研究者の職歴を使って研究機関を埋め込んだ研究がある。
研究活動の分析で扱うデータには、キーワードや要旨などのテキストデータ、共著関係、引用関係といったネットワークデータが多く、埋め込み法が活用され始めている。

% close the loop
自然言語処理の技術として発展した埋め込み法が、複雑ネットワーク研究に新たな学際的な広がりをもたらしている。
人々の社会活動がデジタル世界にシフトする中、膨大な量のテキストデータやネットワークデータが生まれている。

計算社会科学の
コミュニケーション技術の発展や書誌のデジタルアーカイブによって、
大規模社会データを情報技術

大量のテキスト、ネットワークデータが生まれている。

SNSに代表されるコミュニケーション技術の発展と普及によって人々は
ネットワーク科学に新たな流れをもたらしている。
が
最近急速に発展している計算社会学に

複雑ネットワーク研究に学際的な広がりをもたらしている。

ます。この点で特集号の趣旨によく合うかと思います

\
% 謝辞
%------------------
\acknowledgement
%-----------------

\bibliographystyle{unsrt}
\bibliography{main}


\authorbiography{幸,若, ,完,壮}{こう,じゃく,,さだ,もり}{非正会員}{%
 2015年9月××大学大学院工学研究科○○工学専攻△△課程修了.
 同年X月××助手.19XX年X月××となり現在に至る.
 ××の研究に従事.
 ××などの会員.}

%\authorbiography{佐,藤, ,花,子}{さ,とう,,はな,こ}{正会員}{%
%  19XX年X月××大学大学院工学研究科○○工学専攻△△課程修了.
%  同年X月××助手.19XX年X月××となり現在に至る.
%  ××の研究に従事.
%  ××などの会員.}

%\authorbiography{田,中, ,次,郎}{た,なか,,じ,ろう}{正会員}{%
%  19XX年X月××大学大学院工学研究科○○工学専攻△△課程修了.
%  同年X月××助手.19XX年X月××となり現在に至る.
%  ××の研究に従事.
%  ××などの会員.}


\end{document}
%
%% end of exposit.tex
